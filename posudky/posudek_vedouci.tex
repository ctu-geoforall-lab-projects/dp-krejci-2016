\documentclass[czech,11pt,a4paper]{article}
\usepackage[utf8]{inputenc}
\usepackage{a4wide}
\usepackage[pdftex,breaklinks=true,colorlinks=true,urlcolor=blue,
  pagecolor=black,linkcolor=black]{hyperref}
\usepackage[czech]{babel}

\pagestyle{empty}

\begin{document}

\begin{center}
  {\Large --- Posudek vedoucího diplomové práce ---}
\end{center}

\vspace{.2cm}

\noindent \begin{tabular}{rp{.9\textwidth}}
  {\bf Název:} & Processing of vector data using distributed database systems in GIS \\
  {\bf Student:} & Bc. Matěj Krejčí \\
  {\bf Vedoucí:} & Ing. Martin Landa, Ph.D. \\
  {\bf Fakulta:} & Fakulta stavební ČVUT v Praze \\ 
  {\bf Katedra:} & Katedra geomatiky \\
  {\bf Oponent:} & Ing. Jan Pytel, Ph.D. \\
  {\bf Pracoviště oponenta:} & Katedra geomatiky, FSv ČVUT v Praze \\
\end{tabular}

\vspace{1cm}

Zadání původně vzešlo z bakalářské práce, kterou student zpracoval v
roce 2014 na téma \uv{Analýza a vizualizace srážkových dat z
  mikrovlnných telekomunikačních spojů pomocí GIS}. Jejím praktickým
výstupem byla sada nástrojů pro systém GRASS a databáze mikrovlnných
spojů provozovaná na školním serveru geo102. Oba výstupy byly
vytvořeny \uv{na objednávku} pro kolegy z katedry hydrauliky a hydrologie v
rámci řešeného grantu GAČR. Jako databázové uložiště byl zvolen
objektově-relační databázový systém PostgreSQL. Toto řešení se
postupem času s narůstajícím počtem záznamů (aktuálně přes 3 miliardy
záznamů) ukázalo jako nedostačující a nevhodné. V~této době jsme
společně s kolegy uvažovali jak s~těmito daty naložit optimálnějším
způsobem, což nás vedlo přirozeně k problematice tzv. velkých dat
(\uv{big data}). Na základě této úvahy vznikla předkládaná diplomová
práce. \newline

Text práce je napsán v anglickém jazyce. Důraz je kladen především na
problematiku zpracování geografických dat, resp. dat s prostorovou
složkou. Přestože je téma vysoce aktuální a~přínosné, tak nám není
známo, že by se mu věnovaly další práce na našem oboru. Autor popisuje
problematiku velkých dat nejprve obecně. V další části se věnuje
specifikám práce s geografickými daty, dostupných knihoven a
technologii jako celku. Na začátku práce jsme uvažovali o
technologiích, které budou použity pro její praktickou část. V
prvotní fázi student experimentoval se sítí počítačů virtualizovaných
jako kontejnery Docker. Tento postup se jevil jako příliš
komplikovaný, proto byla po konzultaci s kolegou Ing. Janem Pytlem,
Ph.D. zvolena technologie Google Cloud. Google navíc nabízí bezplatný
program pro studenty, který umožňuje jejich technologie a servery
využívat bezplatně po dobu tří měsíců. Této možnosti bylo pro
zpracování práce využito. \newline

Praktická část práce je postavena na dvou datových sadách. První
navazuje na výše zmiňovanou bakalářkou práci. Jde o export z databáze
mikrovlnných spojů. Zde si student vyzkoušel praktickou práci s
platformou Google Cloud a importem velkých dat, která vyexportoval z
lokální databáze PostgreSQL ve formátu CSV (viz příloha A). Tato
datová sada má fixní prostorovou složku (umístění vysílačů se v čase
nemění), dynamické jsou v čase přirozeně naměřené hodnoty. Vzhledem k
tomu není tato sada vhodná pro prostorové analýzy, typicky typu
agregace či zonální statistiky. Bylo tedy třeba najít takovou datovou
sadu, která splňuje co do počtu prvků podmínku velkých dat a zároveň
je její dynamika v čase svázána právě s~prostorovou složkou popisu
geoprvků. Z volně dostupných zdrojů dat padla volba na databázi
OpenStreetMap. Zpracována byla datová sada pro Evropu, která má včetně
historie od jejího vzniku přes 1.3 bilionů prvků. Výsledek tohoto
snažení je popsán v kapitole 3.2. Zde je také demonstrováno propojení
se systém GRASS GIS. S tím souvisí zásadní praktický výstup práce, a
to sada nástrojů umožňující komunikaci mezi systémy GRASS a
technolo\-giemi Hadoop a Hive. Vzniklé nástroje budou dostupné pro
ostatní uživatele systému GRASS v~rámci tzv. Addons. \newline

Zadání práce vzniklo ze studentovi iniciativy, byl schopen se v
relativně krátkém čase zorientovat pro něj v naprosto nové
problematice a rozumně ji uchopit. Na základě toho vznikl praktický
výstup práce, a to nástroje pro propojení systému GRASS a technologií
Hadoop, Hive. Z~těchto nástrojů mohou těžit a dále je rozvíjet i další
uživatelé systému GRASS. Zpracování práce považuji za naprosto
odpovídající vstupním podmínkám a jako vedoucí práce jsem s~jejím
výsledkem spokojen. \newline

Na základě výše uvedeného hodnotím předloženou diplomovou práci
klasifikačním stupněm

\begin{center}
  {\bf --- A (výborně) --- }
\end{center}

\vskip 2cm

\begin{tabular}{lp{.2\textwidth}r}
& & \ldots\ldots\ldots\ldots\ldots\ldots\ldots \\
V~Solanech dne 17. června 2016 & & Ing. Martin Landa, Ph.D. \\
& & Fakulta stavební, ČVUT v Praze \\
\end{tabular}

\end{document}
